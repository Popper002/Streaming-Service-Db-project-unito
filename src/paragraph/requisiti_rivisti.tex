\section{Requisiti rivisti}
Si vuole realizzare una base di dati per un servizio che permette di fare live streaming su vari
argomenti1. Il live streaming (o, più sinteticamente, la live) permette di interagire con il pubblico in
tempo reale grazie a feed video, chat e altro. \newline
Ogni utente può essere spettatore o streamer o entrambi. Gli spettatori possono essere registrati al servizio oppure guardare le live in modo anonimo. \newline
Per registrarsi gli utenti devono indicare nome utente, password, data di nascita, numero di telefono o indirizzo mail. \newline
Gli utenti registrati possono chattare, segure lo streamer, creare dirette , ed hanno accesso ad un portfoglio di bit (moneta virtuale) acquistabile in piattaforma . \newline
Ogni streamer ha un canale, che può essre caratterizzato tramite una descrizione. Per ogni canale è possibile modificare una lista di social associati,un'immaagine del profilo e anche un trailer. \newline
In ogni canale possono esserci live, video (live passate) e clip (video di breve durata). Le live possono anche non diventare video del canale. Ognuno ha un titolo, una durata, appartiente a una categoria e può essere associato per diversi tag. \newline
Per ogni live viene memorizzato il nunero medio di spettatori mentre per i video e le clip il numero di visualizzazioni. \newline
Se uno streamer rispetta determinati parametri di performance (un minimo di 500 minuti trasmessi, una media di tre o più spettatori simultanei,alemno 50 follower) può diventare affiliate. \newline
I viewer possono supportare gli streamer tramite le subscription al loro canale, ottenenendo dei privilegi. \newline

La base di dati deve supportare le seguenti operazioni:
\begin{itemize}
  \item  Una volta al giorno si controllano le condizioni per la qualifica di affiliate.
  \item  Una volta a settimana viene calcolata la classifica degli streamer più seguiti.
\end{itemize}

\section{Requisiti rivisiti in gruppi di frasi omogenee}
\subsection{Fase relativa agli utenti}
\begin{itemize}
  \item Gli utenti si dividono in registrati o anonimi
\end{itemize}
\subsection{Fase relativa a utenti anonimi}
\begin{itemize}
  \item Gli utenti anonimi possono visitare i canali senza doversi per 
  forza regstrare, ma non possono interagire 
\end{itemize}
\subsection{Fase relativa a utenti registrati}
\begin{itemize}
  \item Gli utenti per registrarsi devono:
  \begin{itemize}
    \item Registrarsi fornendo: username,passowrd, email o numero di telefono, data di nascita
  \end{itemize}
  E possono accedere a:
    \begin{itemize}
      \item Canale 
      \item chat privata o publica
      \item portafoglio (per eventuali donazioni)
      \item possono dare \textit{Supporto}\footnote{Supporter} tramite \textit{subscription}
    \end{itemize} 
  Per utenti che creano contenuti invece:
    \begin{itemize}
      \item Il numero di live effettuate 
      \item il numero di minuti trasmessi
      \item il numero medio di spettatori \footnote{Tutti questi dati si possono usare per far diventare uno \textit{streamer affiliate}} %TODO: questo si può vedere anche come vincolo%
    \end{itemize}
\end{itemize}
\subsection{Fase relativa al canale}
Un canale è composto da:
\begin{itemize}
  \item Descrizione
  \item Lista dei social
  \item Immagine del profilo
  \item Trailer 
  \item Live 
  \item Video e clip \footnote{Queste non sono in streaming ma vengono salvate}
  \item Ore di streaming
\end{itemize}
\subsection{Fase relativa alle live}
\begin{itemize}
  \item Possono essere viste da tutti gli utenti
  \item Iniziano a un determinato orario \newline
  Viene memorizzato:
    \begin{itemize}
      \item Il numero medio di spettatori
      \item Chat 
      \item il titolo
    \end{itemize}
    \item A ogni live corrisponde un URL
\end{itemize}
\subsubsection{Fase relativa a una chat}
\begin{itemize}
  \item Solo gli utenti registrati possono accedere alla chat che può essere:
  \begin{itemize}
    \item Privata (Tra utente e utente)
    \item Pubblica (Tra utente e utenti) 
  \end{itemize}
\end{itemize}
\subsection{Fase relativa alle Operazioni}
\begin{itemize}
  \item Registrazione utente
  \item donazioni \footnote{A utenti registrati}
  \item Diventare follower di un canale
  \item Avviare una chat privata tra utenti registrati
\end{itemize}
