\chapter{Revisione dei requisiti}

\section{Fase relativa agli utenti}
\begin{itemize}
  \item Gli utenti si dividono in registrati o anonimi
\end{itemize}
\section{Fase relativa a utenti anonimi}
\begin{itemize}
  \item Gli utenti anonimi possono visitare i canali senza doversi per 
  forza regstrare, ma non possono interagire 
\end{itemize}
\section{Fase relativa a utenti registrati}
\begin{itemize}
  \item Gli utenti per registrarsi devono:
  \begin{itemize}
    \item Registrarsi fornendo: username,passowrd, email o numero di telefono, data di nascita
  \end{itemize}
  E possono accedere a:
    \begin{itemize}
      \item Canale 
      \item chat privata o publica
      \item portafoglio (per eventuali donazioni)
      \item possono diventare \textit{supporter} tramite \textit{subscription}
    \end{itemize} 
  Per utenit che creano contenuti invece:
    \begin{itemize}
      \item Il numero di live effettuate 
      \item il numero di minuti trasmessi
      \item il numero medio di spettatori \footnote{Tutti questi dati si possono usare per far diventare uno \textit{streamer affiliate}} %TODO: questo si può vedere anche come vincolo%
    \end{itemize}
\end{itemize}
\section{Fase relativa al canale}
Un canale è composto da:
\begin{itemize}
  \item Descrizione
  \item Lista dei social
  \item Immagine del profilo
  \item Trailer 
  \item Live 
  \item Video e clip \footnote{Queste non sono in streaming ma vengono salvate}
  \item Ore di streaming
\end{itemize}
\section{Fase relativa alle live}
\begin{itemize}
  \item Possono essere viste da tutti gli utenti
  \item Iniziano a un determinato orario
  Viene memorizzato:
    \begin{itemize}
      \item Il numero medio di spettatori
      \item Chat 
      \item il titolo
    \end{itemize}
    \item A ogni live corrisponde un URL
\end{itemize}
\section{Fase relativa a una chat}
\begin{itemize}
  \item Solo gli utenti registrati possono accedere alla chat
  Può essere:
  \begin{itemize}
    \item Privata (Tra utente e utente)
    \item Pubblica 
  \end{itemize}
\end{itemize}
\section{Fase relativa alle Operazioni}
\begin{itemize}
  \item Registrazione utente
  \item donazioni \footnote{A utenti registrati}
  \item Diventare follower di un canale
  \item Avviare una chat privata tra utenti registrati
\end{itemize}